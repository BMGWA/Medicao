\begin{resumo}
A presença de dívida técnica, ou débito técnico, em uma aplicação pode trazer perdas significativas para as organizações que produzem essas aplicações. Poder conhecer e gerenciar essa dívida técnica é uma estratégia eficiente para se obter um maior controle da dívida na aplicação e poder usar ela em seu favor, se beneficiando melhor de seus ganhos a curto prazo e mantendo as perdas a longo prazo em faixas de interesse da organização. Neste trabalho é proposto uma metodologia de medição usando GQM (\textit{Goal Question Metric}) e dívida técnica em um contexto de contratação de serviços de TI da administração pública federal.

 \vspace{\onelineskip}
    
 \noindent
 \textbf{Palavras-chaves}: dívida técnica. GQM. TI APF. administração públic federal.
\end{resumo}
