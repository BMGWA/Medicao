\chapter[Plano de Medição de TD para o Método Squale]{Plano de Medição de TD para o Método Squale}

Esta sessão trará tópicos relativos ao Planejamento e Definição do GQM que será
proposto para este trabalho de medição de Débito técnico.

\section{Planejamento do GQM}
\subsection{Estabelecer Equipe do GQM}
A equipe do GQM do Projeto de Medições para Dívida Técnica (PMDT) é composta por
cinco integrantes cursandos da Disciplina de Medição e Análise de 2015-2. Estes
foram divididos em papeis dentro do grupo para a divisão de atividades de acordo
com o interesse individual. Tivemos a seguinte divisão de acordo com os respectivos
papeis:
\\


\begin{table}[ht]
\caption{Divisão dos Papeis na Equipe GQM}
\centering
\begin{tabular}{|l*{1}{c}r|}
\hline
Aluno              & Matrícula & Papel \\
\hline
Iago Golçalves & 13/0010219 &   Embasamento Teórico   \\
\hline
Jonathan  Rufino & 10/0107826 &   Coleta de Dados\\
\hline
Lucas Mattiolli & 13/0060364 &   Embasamento Teórico \\
\hline
Tiago Assunção & 13/0051187 &   Definição do GQM  \\
\hline
Wesley Araújo & 13/00392017 &   Planejamento do GQM \\
\hline
\end{tabular}
\label{table:papeisgqm}
\end{table}

Como visto, possuímos quatro papeis para a equipe de GQM do PMDT que são fundamentais
para que juntos, formem o plano de medição de Dívida Técnica. Eles possuem as respectivas
responsabilidades, como podemos ver a seguir:

\begin{itemize}
  \item Embasamento Teórico, papel responsável por assegurar o grupo sobre as decisões
  práticas com solidez acadêmica adiquirida através de anos de pesquisa por pesquisadores,
  alunos, mestre e doutores.

  \item Planejamento do GQM, papel responsável por desenhar e planejar o GQM, estimar
  as equipes, projeto de atuação, área de melhoria, treinar e promover a medição
  no ambiente operacional.

  \item Definição do GQM, papel responsável por concretizar o GQM, definindo os
  Objetivos de Medição, Questões de Medição e, por fim, as métricas. Após toda a
  definição, essa equipe é responsável por validar as métricas propostas de acordo
  com o projeto.

  \item Coleta de Dados, papel de implantar as medições do plano nos sistemas e
  progetos
\end{itemize}

\subsection{Selecionar Área de Melhoria}
\subsection{Selecionar Projetos de Aplicação}
\subsection{Criar Plano de Projeto de Medição}
\subsection{Treinar e Promover}

\section{Definição do GQM}
\subsection{Objetivos de Medição}
\subsection{Produzir Modelos de Processo}
\subsection{Conduzir Entrevistas GQM}
\subsection{Definir Questões}
\subsection{Definir Métricas}
\subsection{Verificar Métricas}
\subsection{Produzir Planos}
