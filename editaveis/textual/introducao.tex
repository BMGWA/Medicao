\chapter[Introdução]{Introdução}

No decorrer do desenvolvimento de um projeto de \textit{software}, podem ocorrer eventos
em que a qualidade do código é negligenciada devido as restrições de custo/tempo,
ou decisões estratégicas para ganhar oportunidades de mercado. O débito técnico
é uma metáfora, proposta por \cite{cunningham}, que faz alusão a esses eventos.

Essa metáfora faz analogia ao débito financeiro, onde os eventos em que a
qualidade do \textit{software} é comprometida para atender a uma restrição ou obter
um benefício a curto prazo, seria como adquirir uma dívida financeira. O não
pagamento dessa dívida acarretará em juros, que é o aumento do esforço para
saudar essa dívida, devido ao aumento da complexidade do software ou mudanças
do projeto no futuro.

Assim como uma dívida financeira, é importante também gerenciar o débito técnico
em um projeto de \textit{software}. De acordo com \cite{mapping}, das atividades de
gerenciamento de débito técnico, as mais estudadas são: Identificação; Medição;
e Pagamento.

Dos trabalhos apresentados até o momento, observam-se propostas para identificação de
débito técnico, como análise estática de código, \cite{siebra}, medição do débito
através de fórmulas que resultam valores financeiros, \cite{principal}, e
propostas de utilização de estratégias como o SQALE para definição e medição de
débito técnico em projetos, definido por \cite{letouzey}.

\cite{schmid} debate sobre os limites dessa metáfora, em outras palavras, até onde
o débito técnico pode ser comparado com o débito financeiro.

Além destes, também não existem estudos sobre a aplicação de débito técnico para
atestar qualidade dos \textit{softwares} entregues por fábricas de \textit{softwares} e outros
fornecedores, bem como sua aplicação na administração pública.


\section{Justificativa da Pesquisa}
Num contexto específico, o Ministério das Comunicações (MC), em parceria com a
Universidade de Brasília (UnB), tem buscado adotar os métodos Ágeis para a
definição de um processo de gestão da contratação de fábrica de \textit{software} para o
desenvolvimento e manutenção de \textit{Software}.
Os trabalhos se aliam a legislação pública federal. A Instrução Normativa 04,
que rege as contratações de Soluções de Tecnologia da Informação no
 Brasil em órgãos governamentais, impõe restrições que devem ser respeitadas por
 parte da contratada para a contratante e vice-versa. Uma delas, o inciso III do
 art. 20, tange sobre a qualidade do código fornecido, afirmando que pode haver
 punição caso os níveis mínimos fixados nos critérios de aceitação não sejam
 cumpridos. Observa-se quão necessário é identificar meios de se verificar a
 qualidade do código fornecido.

\section{Objetivo Geral}
 Este trabalho objetiva propor uma estratégia de verificação da qualidade dos
 códigos de \textit{softwares} entregues pelas fábricas de \textit{software}, para o Ministério
 das Comunicações, pela análise da quantidade de débito técnico a ser identificado
 nas aplicações entregues.

 O trabalho está organizado em cinco seções. Na Seção II, apresenta-se o
 referencial teórico, com conceito de débito técnico, medição e análise e GQM
 além de algumas técnicas
 apresentadas na literatura, e pequena contextualização da Contratação de
 Serviços de TI na Administração Pública Federal. Na Seção III apresentam-se
 os materiais e métodos, onde são abordados a metodologia utilizada bem como o
 objeto de estudo, o Ministério das Comunicações e seus processos. Na Seção IV
 apresenta-se a proposta de medição de débito técnico utilizando GQM para o objeto de estudo.
 Finalizando, na Seção V, as Considerações finais e Trabalhos futuros.
