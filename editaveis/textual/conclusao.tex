\chapter[Conclusão]{Conclusão}
A medição de Débito Técnico pode trazer muitas informações sobre os benefícios e custo de um software ao 
se adotar se deve ou não haver tal dívida, nos trazendo resultados do impacto, sendo ele negativo ou positivo. Através do GQM, é possível fazer indagações sobre o Débito Técnico para que as medições sejam mais precisas, satisfazendo o objetivo das medições.


Softwares vindos de terceiros pode ser mais difícil de ter as medições por conta de já ter um software totalmente pronto ou com determinada funcionalidade pronta, enquanto a medição em desenvolvimento pode já entregar um projeto em que se sabe as informações da Dívida Técnica com antecedência.


Através desse trabalho, é possível criar formas automatizadas das medições de Dívidas Técnicas, baseadas no GQM, com o fim de se ter os dados com mais agilidade no processo de desenvolvimento do projeto ou com ele completo.

Um trabalho futuro é aliar os resultados das métricas propostas neste trabalho à fórmula de medição de débito técnico SQALE. 