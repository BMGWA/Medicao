\chapter[Referencial Teórico]{Referencial Teórico}

\section{Débito Técnico}
Débito técnico é uma metáfora proposta por \cite{cunningham} onde, uma dívida é
criada quando a qualidade do código é deixada de lado para acelerar o processo
de desenvolvimento, até o momento que esta dívida é quitada. O preço do pagamento
da dívida, geralmente feita através de refatoração,  aumenta gradualmente com
a evolução do projeto.

\cite{oliveira} afirma que se pode identificar vários atributos em um Débito
Técnico, sendo eles:

\begin{itemize}
  \item Reason, motivo pelo qual o débito foi adquirido;
  \item Benefíts, o lucro ou ponto positivo que a escolha de optar pelo débito traz;
  \item Result, o que esse débito realmente possibilitou para a organização;
  \item Principal, representa o preço que o pagamento do débito possui no momento;
  \item Interest, preço de pagamento do débito depois de um dado tempo;
  \item Returns, quanto a organização lucrou com a decisão de ter optado pelo Débito Técnico.
\end{itemize}


De acordo com \cite{mapping}, há um debate sobre a importância de gerenciar o
Débito Técnico, pois este é muito importante para a equipe de negócios,
complementando que nem sempre este é ruim, e que podem trazer benefícios
que são maiores do que a perca monetária causada pelo juros.

\cite{mapping} identificam que o Débito Técnico tem atividades que podem ser classificadas.
Ela lista oito tipos de Débito Técnico, sendo eles:

\begin{itemize}
  \item Identificação;
  \item Medição;
  \item Priorização;
  \item Prevenção;
  \item Monitoramento;
  \item Pagamento;
  \item Representação/Documentação;
  \item Comunicação;
\end{itemize}

Quando se trata de Débito Técnico, existem modelos de identificação e medição de
código. O autor \cite{eisenberg}, executa uma análise estática e utiliza o
resultado para estimar o quanto custará para sanar todo ou parte do Débito
Técnico existente em um projeto.



\subsection{Método SQALE para avaliação de Débito Técnico}
Letouzey [5] propõe a utilização do método SQALE (Software Quality Assessment
based on Lifecycle Expectations) para análise do débito técnico. O método de
Letouzey propõe um Modelo de Qualidade (SQALE Quality Model) e um Modelo de
Análise (SQALE Analysis Model) para verificação das violações de código.

O Modelo de Qualidade do SQALE constitui de uma matriz hierárquica de 3 níveis:
Características, Sub-Caracteristicas e Requisitos. Numa estratégia de análise de
débito técnico, pode ser definido que qualquer não-conformidade do código com os
requisitos caracteriza débito técnico. Por isso, os Requisitos devem ser atômicos,
não-ambíguos, não-redundantes, justificáveis, aceitáveis, implementáveis e
verificáveis.

Assim, basta utilizar uma ferramenta que consiga avaliar o código com relação aos Requisitos estabelecidos e será possível identificar o débito técnico na aplicação.
O Modelo de Análise faz uso de um índice de remediação para avaliar os impactos das violações identificadas. Esse índice de remediação faz uma relação entre os Requisitos do Modelo de Qualidade, com uma remediação associada a cada violação desses requisitos. Por fim, é estipulado também uma função de remediação, que associa um tempo para aplicar aquela remediação.

\subsection{Identificação de Débito Técnico}
\subsection{Medição de Débito Técnico}

\section{Contratação na Administração Pública Federal Brasileira}
